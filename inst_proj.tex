\documentclass[a4paper, 12pt]{article}
\usepackage{fontspec} 
\usepackage[utf8]{inputenc}
\usepackage[russian,english]{babel}
\setmainfont{Times New Roman} % для компиляции теперь нужно использовать XeLaTex 

\usepackage[paper=a4paper,top=13.5mm, bottom=13.5mm,left=16.5mm,right=13.5mm,includefoot]{geometry}

\usepackage{dsfont}
\usepackage{graphicx} 
\graphicspath{{images/}} 
\setlength\fboxsep{3pt} 
\setlength\fboxrule{1pt} 
\usepackage{wrapfig} 
\usepackage{amsmath}
\usepackage{amsthm}
\renewcommand{\proofname}{Доказательство}

\DeclareMathOperator*{\argmin}{arg\,min}


\theoremstyle{definition}
\newtheorem{definition}{Определение}[section]


\theoremstyle{plain}
\newtheorem{lemma}{Лемма}[section]
\newtheorem{svoistvo}{Свойство}[section]
\newtheorem{theorem}{Теорема}[section]


\title{Исследовательский проект по инстуциональной экономике}
\date{\today}

\begin{document}
\maketitle

\noindent{\bf Аннотация }-- ...

\section{Введение}
Мы устверждаем, что в существуют рынки на которых присутствует сразу несколько игроков со стороны спроса на труд в одном регионе. 
\section{Обзор литературы}

\section{Модель}

Многие статьи рассматривают академический рынок труда в рамках одной местности (изменить этот слово), как монопсонию (ссылки на статьи). Мы рассмотрим модель олигпсонического рынка труда в рамках которой у профессора будет выбор либо остаться в своем регионе и пойти в один из университетов, либо уехать на понести издержки переезда. Модель будет основана на статье ( бла бла). 

Пусть существует рынок в рамках одного города, где присутствует $n$ университетов одного уровня, предъявляющие спрос на одних и тех же академических работников (смищно). Каждый университет может предлагать разную зарплату разным профессорам. Зарплату предложенную $i$ университетом мы будет обозначать $w_i$. В случае, если профессор отказывается, то университет несет издержки равные зарплате профессора с конкурентного рынка равные $w_m$ (переписать этот кусок, потому что копи паст) (Возможно вставить сначала предпосылки относительно поведения профессора). Университет минимизирует свои ожидаемые изержки: 
\[
EC(w_i) = p(w_m, w_i, m, w_{j \neq i})w_i + (1 - p(w_m, w_i, m, w_{j \neq i}))w_m
\]




\section{Кейс}

\section{Заключение}




\end{document}



